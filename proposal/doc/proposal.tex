\documentclass[a4paper, 11pt]{article}

\usepackage{minted}

\usepackage[a4paper, margin=2cm]{geometry}


\title{Master Thesis Proposal:\\\vspace{0.5cm}{\Huge Functional Property Discovery and Corresponding Test Generation in Haskell}}
\author{Tom Sydney Kerckhove}
\begin{document}
\maketitle

\section{Introduction and Motivation}

Correctness of code is currently too expensive.
Current options, in order of decreasing rigour and decreasing cost, include formal methods, testing and having users file bug reports after failures have already ocurred.

When formal methods are too expensive, testing is the most reliable way to ensure that a software system does what it should.
Traditional unit tests have two problems.
The first problem is that unit tests can only be used to show the presence of failures, but can never show the absence of failures.
The second problem is that unit tests require a programmer to provide many examples for the unit tests to be of real value in providing confidence in the implementation.
This requires great discipline on the programmer's part.
Traditional unit tests are still too expensive.
As a result, programmers will all too often omit tests.

Property-based testing \cite{QuickCheck} can probabilistically solve the first problem by automatically generating examples instead of having the programmer think of them.
As such, property-based testing can show failures that the programmers did not anticipate and that the programmer would not have tested for.
However, it exacerbates the second problem.
Now programmers have to come up with mathematical properties of their code.
The transaction costs of writing property tests are even higher than for unit tests.
As a result, however great their benefits, property tests are barely used.

Automatic property tests could resolve the second problem of traditional unit tests.
If properties could automatically be discovered, and their corresponding property tests automatically generated, then programmers would not have to think of any properties, but only decide to enforce the properties.
Relieving a programmer from having to think of examples or properties would make it much cheaper to test software.

Previous work \cite{QuickSpec} has explored automatic discovery of equational properties.
These are properties of the form \mintinline{Haskell}{\x -> f x == g x}, e.g. \mintinline{Haskell}{\xs -> reverse (reverse xs) == xs}.
Equational properties, while certainly useful, fall short in describing real workings of even simple programs.
For example, there is no way to specify the correctness of a sorting algorithm using an equational property.

The first goal of this thesis is to explore automatic discovery of general properties.
General properties allow for more expressive properties such as the correctness of a sorting algorithm.\footnote{The property \mintinline{Haskell}{\xs -> sort xs `isPermutationOf` xs && ordered ss}, for example, serves to assess the correctness of a sorting algorithm, but can not be expressed as an equational property.}
The second goal is to, once properties have been discovered, generate the code necessary to test that these properties hold.

\newpage

\section{Assignment}
\subsection{Objective}
Develop a tool that can inspect Haskell code and suggest properties of that code to the programmer.
After the programmer has selected the appropriate properties, the tool generates the necessary code to test these properties.

\subsection{Tasks}

\begin{itemize}
    \item Study and analyse the relevant literature.
    \item Propose a property discovery mechanism.
    \item Implement property discovery.
    \item Implement test code generation.
    \item Assess the feasibility of the property discovery mechanism for increasingly large input programs.
    \item (Optional) Assess the relative advantage of general property testing over equational property testing in a quantitative manner.
    \item Write the final report and prepare a presentation
\end{itemize}


\subsection{Deliverables}

\begin{itemize}
    \item Final report
    \item Tool code and instructions on how to use it.
    \item Presentation
\end{itemize}

\bibliographystyle{plain}
\bibliography{proposal}


\end{document}
