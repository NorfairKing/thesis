\documentclass[a4paper, 11pt]{article}

\usepackage{minted}

\title{Thesis Proposal: Functional Property Discovery and Test Generation}
\author{Tom Sydney Kerckhove}
\begin{document}
\maketitle

\section{Introduction and Motivation}

When verification is too expensive, testing is the most reliable way to ensure that a software system does what it should.
While traditional unit tests can be used to show failures, they are insufficient to show the absence of failures.
Traditional unit tests have the disadvantage that the programmer has to make up many examples for the unit tests to be of real value in providing confidence in the implementation.
Property-based testing \cite{QuickCheck} allows for more rigorous testing but requires the programmer to come up with properties to test.
While property-based testing can give probabilistic confidence that code behaves well, and therein improves upon traditional unit testing, it has the same problem that traditional unit testing has.
Tests are still conceived as simply to expensive to write, in terms of developer time.
The transaction cost of writing tests, even though the tests may be a great time investment, are often high enough to prevent programmers performing rigorous tests.

Property-based testing takes care of the 'rigour' aspect of the testing problem, but programmers still have to think of properties.
Automatic discovery of properties could take of the 'transaction costs' aspect of the testing problem.
Previous work\cite{QuickSpec} has explored automatic discovery of equational properties.
Equational properties, while certainly useful, fall short in describing real workings of even simple programs.
For example, there is no way to specify the correctness of a sorting algorithm using an equational property.

The first goal of this thesis is to explore automatic discovery of generalised properties.
General properties allow for much more expressive properties such as the correctness of a sorting algorithm.\footnote{\mintinline{Haskell}{\xs -> sort xs `isPermutationOf` xs && ordered ss}, for example.}
The second goal is to, once properties have been discovered, generate the code necessary to test that these properties hold.

\section{Assignment}
\subsection{Objective}
Develop a tool that can inspect Haskell code and suggest properties of that code to the programmer.
After the programmer has selected the appropriate properties, the tool should be able to generate the necessary code to test these properties.

\subsection{Tasks}

\begin{itemize}
    \item Study and analyse the relevant literature.
    \item Propose a property discovery mechanism.
    \item Implement property discovery.
    \item Implement test code generation.
    \item Assess the feasibility of the property discovery mechanism for increasingly large input programs.
    \item (Optional) Assess the relative advantage of general property testing over equational property testing in a quantitative manner.
    \item Write the final report and prepare a presentation
\end{itemize}


\subsection{Deliverables}

\begin{itemize}
    \item Final report
    \item Tool code and instructions on how to use it.
    \item Presentation
\end{itemize}

\bibliographystyle{plain}
\bibliography{proposal}


\end{document}
